% Options for packages loaded elsewhere
\PassOptionsToPackage{unicode}{hyperref}
\PassOptionsToPackage{hyphens}{url}
%
\documentclass[
]{article}
\usepackage{lmodern}
\usepackage{amssymb,amsmath}
\usepackage{ifxetex,ifluatex}
\ifnum 0\ifxetex 1\fi\ifluatex 1\fi=0 % if pdftex
  \usepackage[T1]{fontenc}
  \usepackage[utf8]{inputenc}
  \usepackage{textcomp} % provide euro and other symbols
\else % if luatex or xetex
  \usepackage{unicode-math}
  \defaultfontfeatures{Scale=MatchLowercase}
  \defaultfontfeatures[\rmfamily]{Ligatures=TeX,Scale=1}
\fi
% Use upquote if available, for straight quotes in verbatim environments
\IfFileExists{upquote.sty}{\usepackage{upquote}}{}
\IfFileExists{microtype.sty}{% use microtype if available
  \usepackage[]{microtype}
  \UseMicrotypeSet[protrusion]{basicmath} % disable protrusion for tt fonts
}{}
\makeatletter
\@ifundefined{KOMAClassName}{% if non-KOMA class
  \IfFileExists{parskip.sty}{%
    \usepackage{parskip}
  }{% else
    \setlength{\parindent}{0pt}
    \setlength{\parskip}{6pt plus 2pt minus 1pt}}
}{% if KOMA class
  \KOMAoptions{parskip=half}}
\makeatother
\usepackage{xcolor}
\IfFileExists{xurl.sty}{\usepackage{xurl}}{} % add URL line breaks if available
\IfFileExists{bookmark.sty}{\usepackage{bookmark}}{\usepackage{hyperref}}
\hypersetup{
  hidelinks,
  pdfcreator={LaTeX via pandoc}}
\urlstyle{same} % disable monospaced font for URLs
\setlength{\emergencystretch}{3em} % prevent overfull lines
\providecommand{\tightlist}{%
  \setlength{\itemsep}{0pt}\setlength{\parskip}{0pt}}
\setcounter{secnumdepth}{-\maxdimen} % remove section numbering
\ifluatex
  \usepackage{selnolig}  % disable illegal ligatures
\fi

\author{}
\date{}

\begin{document}

\hypertarget{nysg-controller-construction-guide}{%
\section{NYSG Controller Construction
Guide}\label{nysg-controller-construction-guide}}

\hypertarget{summary}{%
\subsection{Summary}\label{summary}}

This guide is meant as a way for you to build your own controller.
Follow the instructions carefully and use the Jupyter Notebooks for
reference when you build the software controller. You can still use the
system without the controller you build. Our controller also works as
well, which you can use to compare functionality with.

\hypertarget{materials}{%
\subsection{Materials}\label{materials}}

You only need one file: \textbf{student\_controller\_main.py}

\textbf{\emph{DO NOT MAKE EDITS IN ANY OTHER FILES.}}

\textbf{For Reference} Please reference controller\_main.py if you ever
get stuck. Also reference the jupyter notebook in the folder named
Jupyter Notebook.

\textbf{\emph{AGAIN, DO NOT MAKE EDITS IN ANY OTHER FILES.}}

\hypertarget{the-code-you-will-write.}{%
\subsection{The Code You Will Write.}\label{the-code-you-will-write.}}

All the code you will write are located in the sections marked at the
beginning with ----------------------- EDIT BELOW HERE
-----------------------

and ending with ----------------------- EDIT ABOVE HERE
-----------------------

\textbf{\emph{DO NOT MAKE EDITS IN ANY OTHER AREAS OF THIS FILE.}}
\textbf{\emph{DO NOT MAKE EDITS IN ANY ABOVE OR BELOW THESE DEMARCATED
SECTIONS OF CODE.}}

There will also be TODO flags inside the code portions where you are to
fill in the code and instructions explaning what you need to do.

\hypertarget{prequisites-and-skills-required}{%
\subsection{Prequisites and Skills
Required}\label{prequisites-and-skills-required}}

\begin{enumerate}
\def\labelenumi{\arabic{enumi}.}
\tightlist
\item
  Review and Work through Jupyter Notebooks

  \begin{enumerate}
  \def\labelenumii{\arabic{enumii}.}
  \tightlist
  \item
    Learn Assignment and variables
  \item
    Learn about different data types

    \begin{enumerate}
    \def\labelenumiii{\arabic{enumiii}.}
    \tightlist
    \item
      strings
    \item
      lists
    \item
      ints
    \item
      floats
    \item
      booleans
    \end{enumerate}
  \item
    Learn about control flow

    \begin{enumerate}
    \def\labelenumiii{\arabic{enumiii}.}
    \tightlist
    \item
      if statements
    \item
      for loops
    \item
      while loops
    \end{enumerate}
  \item
    Learn Function Calls
  \item
    Learn how to use external packages and libraries / modules
  \item
    Learn about classes
  \item
    Learn how to debug
  \end{enumerate}
\item
  Patience to figure out a solution
\item
  Ability to reason with code
\end{enumerate}

If you are familiar with programming and with Python, you can consider
the prerequiste part 1 to be satisfied.

\hypertarget{solutions}{%
\subsection{Solutions}\label{solutions}}

Solutions are there if you get stuck in
README-Controller-Main-Solutions.md

\hypertarget{agenda}{%
\subsection{Agenda}\label{agenda}}

There are 8 code chunks you need to fill in for the
student\_controller\_main.py file. You can find these by using CONTROL-F
or ctrl-F or command-F to search for TODO's, which mark the sections
that you are to do complete and fill in.

Here are the descriptions of the tasks. 1. Finish function main() by
filling in one line of code. You need to initialize all the sensors and
peripherals, which requires calling function init(). Set the result of
init to a variable named init\_dict, which is a dictionary holding all
the sensor and peripherals. 2. Let us create the sensors that we need
for this project.

\begin{verbatim}
We need a light sensor, temperature-humidity sensor and moisture sensor.
You can create these objects with the constructors: LightSensor, TempHumiditySensor, MoistureSensor,
To get more information for each constructior, read the module code in
folder Controller and looking in file sensor_class.py

For the light sensor, use the i2c_channel that was created before this section
of code. Similarly, use the i2c_channel for the TempHumidity Sensor.
The Moisture Sensor needs to arguments to be called.
Save the sensors to variables named light_sensor, temp_humid_sensor and moisture_sensor
respectively.
\end{verbatim}

\begin{enumerate}
\def\labelenumi{\arabic{enumi}.}
\setcounter{enumi}{2}
\item
  Let us create the peripherals for this project.

  We need a Solenoid Valve, Heat Pad, Fan and Plant Light. These are
  given by the constructors SolenoidValve, HeatPad, Fan, and PlantLight.
  Each of these takes in a pin number. The pins are: Solenoid Valve:
  pin\_constants.VALVE Heat Pad: pin\_constants.HEAT, Fan:
  pin\_constants.VENT, PlantLight: pin\_constants.LED

  Further, each of the constructors takes in a burst time. Give them all
  20 second burst times. The burst time is an integer.

  The first argument to each constructor is the pin number, while the
  second is the burst time. There are no other arguments.

  Finally, save eacj peripheral to its own variable. The cariables
  should be named, respectively, valve, heat, fan and light.
\item
  We need to be able to read data from the Interface files. We need to
  reading in 5 items: the manual control status, the manual actions, the
  email settings the pwm settings and the frequency settings.

  TO read in data, call the function load\_data. Because load\_data is
  in module pin\_constants, you will have to us the module with a dot
  notation: pin\_constants.load\_data.

  Load\_data takes in one argument, the path of the interface file you
  want to read. When you can in load\_data, load\_data returns to you
  the read in data.

  First, read in manual\_control\_path and save the results of the
  function to a variable named manual\_control.

  Second, read in manual\_actions\_path and save the results of the
  function to a variable named manual\_results.

  Next, read in email\_settings\_path and save the results of the
  function to a variable named email\_settings.

  Then, read in pwm\_settings\_path and save the results of the function
  to a variable named pwm\_settings.

  Finally, read in freq\_settings\_path and save the results of the
  function to a variable named freq\_settings.

  Remember, there are five variables you need to create via 5 function
  calls.
\item
\begin{verbatim}
We need to convert the healthy light integer into a string. We do it by       comparing the value of healthy light. 
\end{verbatim}

  If healthy light is greater than or equal to 4, return the string
  ``Full Sun''

  If healthy light is greater or equal to 3 and less than 4, return
  ``Part Sun''

  If healthy light is greater or equal to 2 and less than 3, return
  ``Part Sun''

  If healthy light is greater or equal to 1 and less than 2, return
  ``Part Shade''

  If healthy light is greater or equal to 0 and less than 1, return
  ``Full Sun''

  Remember how to use if statements, elif statements and booleans.

  Do not add in an else clause.
\item
\begin{verbatim}
    We need to convert action to a string. Action can be an integer between 0 and 4 inclusive, or the string "low" or "high".capitalize

 If action is 0 or it is the string "off", return the string "off".

 If action is 1 return the string "big_decrease".

 If action is 2 or it is the string "low", return the string "low".

 If action is 3 return the string "small_increase".

 If action is 4 or it is the string "high", return the string "high".

 Use your boolean comparison equality operation, == , the double equals,
 as well as if-elif statements. Also return strings inside the if statements.
\end{verbatim}
\item
  We want you to complete the headers of the while loop. In this branch
  of the if statement, we want the code to runforever. We can use a
  while loop with true as the guard. Write in the guard. Then paste in
  the given code below to call the function to run inside the while
  loop. You do not need to know what await does. Just paste the code
  indented inside the while loop body.

\begin{verbatim}
 Given code:
 await one_cycle(init_dict, manual_control_path, manual_actions_path, email_settings_path, pwm_settings_path, freq_settings_path, sensor_log_path, ml_action_log,
                 alert_log, max_log_size, interval)
\end{verbatim}
\item
\begin{verbatim}
    In this branch of the if statement, we want the code to for a
 certain number of iterations, max_iter, specifically. max_iter is an integer
 representing how many iterations to run. We can use a for loop.
  We want you to fill in the for loop guard. Create an iteration variable.
 Use range to make sure you iterate for max_iter iterations.
 Paste in the given code below to call the function to run inside the for loop.
 You do not need to know what await does. Just paste the code indented
 inside the for loop body.

 Given code:
 await one_cycle(init_dict, manual_control_path, manual_actions_path, email_settings_path, pwm_settings_path, freq_settings_path, sensor_log_path, ml_action_log,
                   alert_log, max_log_size, interval)
\end{verbatim}
\end{enumerate}

\hypertarget{testing}{%
\subsection{Testing}\label{testing}}

To test, run ./start-system.sh

The controller and UI should start. Follow instructions to open the UI
and see the results.

\end{document}
