% Options for packages loaded elsewhere
\PassOptionsToPackage{unicode}{hyperref}
\PassOptionsToPackage{hyphens}{url}
%
\documentclass[
]{article}
\usepackage{lmodern}
\usepackage{amssymb,amsmath}
\usepackage{ifxetex,ifluatex}
\ifnum 0\ifxetex 1\fi\ifluatex 1\fi=0 % if pdftex
  \usepackage[T1]{fontenc}
  \usepackage[utf8]{inputenc}
  \usepackage{textcomp} % provide euro and other symbols
\else % if luatex or xetex
  \usepackage{unicode-math}
  \defaultfontfeatures{Scale=MatchLowercase}
  \defaultfontfeatures[\rmfamily]{Ligatures=TeX,Scale=1}
\fi
% Use upquote if available, for straight quotes in verbatim environments
\IfFileExists{upquote.sty}{\usepackage{upquote}}{}
\IfFileExists{microtype.sty}{% use microtype if available
  \usepackage[]{microtype}
  \UseMicrotypeSet[protrusion]{basicmath} % disable protrusion for tt fonts
}{}
\makeatletter
\@ifundefined{KOMAClassName}{% if non-KOMA class
  \IfFileExists{parskip.sty}{%
    \usepackage{parskip}
  }{% else
    \setlength{\parindent}{0pt}
    \setlength{\parskip}{6pt plus 2pt minus 1pt}}
}{% if KOMA class
  \KOMAoptions{parskip=half}}
\makeatother
\usepackage{xcolor}
\IfFileExists{xurl.sty}{\usepackage{xurl}}{} % add URL line breaks if available
\IfFileExists{bookmark.sty}{\usepackage{bookmark}}{\usepackage{hyperref}}
\hypersetup{
  hidelinks,
  pdfcreator={LaTeX via pandoc}}
\urlstyle{same} % disable monospaced font for URLs
\setlength{\emergencystretch}{3em} % prevent overfull lines
\providecommand{\tightlist}{%
  \setlength{\itemsep}{0pt}\setlength{\parskip}{0pt}}
\setcounter{secnumdepth}{-\maxdimen} % remove section numbering
\ifluatex
  \usepackage{selnolig}  % disable illegal ligatures
\fi

\author{}
\date{}

\begin{document}

\hypertarget{nysg}{%
\section{NYSG}\label{nysg}}

\hypertarget{interface-files}{%
\subsection{Interface Files}\label{interface-files}}

Interface files serve as the exchange and storage medium for information
between software subsystems. In this project, we chose to make use of
the JSON format (as opposed to .txt or databases) for simplicity and
ease of use for the end-users (high school students). THese files are
also meant to provide configurability to the user. Changes to the
information in these files will propagate throughout the entire software
system.

The files are read from in both the controller\_main.py and also in the
settings page, analysis page and dashboard pages of the UI. The settings
page also writes back data to the interface files, so that data can be
updated. This shows how the data can persist and be updated and changed.
Essentally, the interface files are a memory medium, rather like RAM or
a database, excepy we structure it by JSON for simplicity in this small
scale project.

\hypertarget{healthy-levels-healthy_levels.json}{%
\subsubsection{Healthy Levels
(healthy\_levels.json)}\label{healthy-levels-healthy_levels.json}}

This file stores healthy levels as specified by the user in the Settings
page of the UI. Each level is represented by a bucket number
corresponding to bucket in value\_buckets.json. These levels serve as
goals onto which the machine learning algorithm will seek to optimize.

\hypertarget{healthy-levels-by-profile-healthy_levels_by_profile.json}{%
\subsubsection{Healthy Levels By Profile
(healthy\_levels\_by\_profile.json)}\label{healthy-levels-by-profile-healthy_levels_by_profile.json}}

This file maps profiles to sets of healthy levels. These profiles will
appear in the drop down menu on the Settings page of the UI as
pre-configured plant profiles.

\hypertarget{healthy-levels-by-profile-sample-healthy_levels_by_profile_sample.json}{%
\subsubsection{Healthy Levels By Profile SAMPLE
(healthy\_levels\_by\_profile\_SAMPLE.json)}\label{healthy-levels-by-profile-sample-healthy_levels_by_profile_sample.json}}

This file maps profiles to sets of healthy levels. These profiles will
appear in the drop down menu on the Settings page of the UI as
pre-configured plant profiles. \textbf{THIS IS A SAMPLE FILE} that
allows the user to have a basic idea of what they will see in the user
interface and is not meant to be used except as a sample or default
method.

\hypertarget{log-log.json}{%
\subsubsection{Log (log.json)}\label{log-log.json}}

This file holds information pertaining to the updates of the system. It
contains one record for each update, labeled with the datetime of that
update. Inside of these records, information on the sensor readings is
stored, as well as the actions that were taken during that update. All
values are presented on a scale from 0-5.9. There are different keys:
sunlight, temperature, humidity, soil\_moisture, water\_action,
heat\_action, fan\_action and light\_action.

\hypertarget{backup-log-log_backup1.json}{%
\subsubsection{Backup Log
(log\_backup1.json)}\label{backup-log-log_backup1.json}}

This file holds information pertaining to the updates of the system. It
contains one record for each update, labeled with the datetime of that
update. Inside of these records, information on the sensor readings is
stored, as well as the actions that were taken during that update. All
values are presented on a scale from 0-5.9. There are different keys:
sunlight, temperature, humidity, soil\_moisture, water\_action,
heat\_action, fan\_action and light\_action.

\textbf{THis file is a backup file} which means when the user clears all
data on the advanced settings page, all data is deleted in log.json and
exported to log\_backup1.json. In the case you clear data agin from log,
log\_backup1 is permanently cleared forever!

\hypertarget{sample-log-log_sample.json}{%
\subsubsection{Sample Log
(log\_SAMPLE.json)}\label{sample-log-log_sample.json}}

This file holds information pertaining to the updates of the system. It
contains one record for each update, labeled with the datetime of that
update. Inside of these records, information on the sensor readings is
stored, as well as the actions that were taken during that update. All
values are presented on a scale from 0-5.9. There are different keys:
sunlight, temperature, humidity, soil\_moisture, water\_action,
heat\_action, fan\_action and light\_action.

\textbf{This file is a sample file}. It is mean t as a sample for the UI
for the user to visualize how data would show up and be seen. It is not
meant for production use.

\hypertarget{manual-actions-manual_actions.json}{%
\subsubsection{Manual Actions
(manual\_actions.json)}\label{manual-actions-manual_actions.json}}

This file contains the last-submitted set of manual actions. When the
system is in ``Manual'' mode (as set in the Settings page of the UI),
the system will take these actions on each update. These actions are set
in the user interface file directly and are reflected in the JSON files.

Actions can be low, high and off, and it is for the water, fan, heater
and light peripherals.

\hypertarget{mode-mode.json}{%
\subsubsection{Mode (mode.json)}\label{mode-mode.json}}

This file contains the mode that the system is currently in. The mode is
either manual or machine learning. it is read in in the controller\_main
to decide what happens in the update cycle; either the UI manual actions
are taken or the ML actions for that cycle are called and taken. The
mode is itself written tok by the user in the advanced settings tab of
the UI.

\hypertarget{plant-profile-plant_profile.json}{%
\subsubsection{Plant Profile
(plant\_profile.json)}\label{plant-profile-plant_profile.json}}

This file contains the current plant profile (as set in the Settings
page of the UI). The profile maps to one of the profiles in
healthy\_levels\_by\_profile.json. The profile could be custom, or
someother preloaded one like ``tomatoes'' that the user could not
change.

\hypertarget{value-buckets-value_buckets.json}{%
\subsubsection{Value Buckets
(value\_buckets.json)}\label{value-buckets-value_buckets.json}}

This file maps buckets to nominal values. Buckets are used to interpret
data throughout the system. For each environment variable (temperature,
humidity, soil moisture, sunlight) there are 5 buckets. Each bucket is
associated with a label, and low value, and a high value. The label is
used in the UI to associate the bucket with relatively familiar concepts
(cold/warn/hot), the low value is the floor of nominal values that would
be associated with that bucket, and the high value is the ceiling of
nominal values that would be associated with that bucket.

\hypertarget{ml-training-actions.json}{%
\subsubsection{ML Training
(actions.json)}\label{ml-training-actions.json}}

This file contains all training operations that the ML can tsake over 64
time steps to check what the ML does and to train it properly. For each
time step, there is a decision listed for the water, fan and heater, in
terms of high, and low and off status. There are 64 time steps specified
in total.

\hypertarget{alert-alert_log.json}{%
\subsubsection{Alert (alert\_log.json)}\label{alert-alert_log.json}}

This file contains all the alerts that the system alerts by time to the
water level that is recorded.

\hypertarget{default-home-address-defaul_home_address.json}{%
\subsubsection{Default Home Address
(defaul\_home\_address.json)}\label{default-home-address-defaul_home_address.json}}

This file contains the defaul home address in Suitville Maryland for the
National Weather Service Address reading. This file \textbf{MUST NEVER
BE CHANGED} because it is a default setting to display should the user
give a faulty or incorrectly entered home address.

\hypertarget{email-settings-email_settings.json}{%
\subsubsection{Email Settings
(email\_settings.json)}\label{email-settings-email_settings.json}}

This file contains the user email settings that is in terms of rate of
email sending and the detail of the email. This file is changed by the
advanced settings of the user page in the UI settings. You can set
minutely, hourly and daily updates of emails and high and low settings
of detail where high is all information and extreme data while low just
shows extreme data. The user changes these settings in the advanced
settings.

\hypertarget{frequency-freq_settings.json}{%
\subsubsection{Frequency
(freq\_settings.json)}\label{frequency-freq_settings.json}}

This file contains the user frequency settings for the pwm frequency
settings for the fan and the plant light. The keys are fan and light and
the inner keys is frequency. The user changes the frequency individually
for light and fan on the UI advanced settings page. The frequency can be
cahanged in units of 10 from 0 to 500 hertz.

\hypertarget{germination-germination.json}{%
\subsubsection{Germination
(germination.json)}\label{germination-germination.json}}

This file contains the germination start and end date with day, month
and year. It holds start and end dates. The start \textbf{can} be equal
to the end date and but it can \textbf{never} be after the end date.
Note that the UI disallows the satrt being after the end. Also note that
the start can be in the future and the end in the future. The start date
could also be in the past. Both start and end are specified in the UI
advanced settings where the user manually sets six fields of day, month
and year for the start and end dates.

\hypertarget{home-address-home_address.json}{%
\subsubsection{Home Address
(home\_address.json)}\label{home-address-home_address.json}}

This file holds the home address for the user including street address,
city, state and zip code. It must be a valid american street address. It
could be an invalid address becaudse the UI does nto check for
invalidity. It is iset in the UI settings advanced settings tab and the
user types in street address and city and zip code and state is chosen
in a drop down. This influences what weather forecast is received by the
user in the weather page. The user should set a local address for local
weather forecasts.

\hypertarget{constants-interface_constants.py}{%
\subsubsection{Constants
(interface\_constants.py)}\label{constants-interface_constants.py}}

This contains constants used for the interface files like log path,
germination path and other paths.

\hypertarget{update-interval-interval_settings.json}{%
\subsubsection{Update Interval
(interval\_settings.json)}\label{update-interval-interval_settings.json}}

This file contains the update interval in \textbf{seconds}. It could be
any value in 60, 180 and 300 seconds. This file can be changed in the
User interface advanced settings tab. This file interacts with the
controller and tells the controller how long to await for the next
update cycle in an await sleep call.

\hypertarget{ml-log-ml_action_log.json}{%
\subsubsection{ML Log
(ml\_action\_log.json)}\label{ml-log-ml_action_log.json}}

This file contains the actions of the ML taken in the last cycle
including keys of water, fan heat, light and expected reward. It is
logged from controller\_main and it cannot be changed or deleted from
the UI.

\hypertarget{duty-cycles-pwm_settings.json}{%
\subsubsection{Duty Cycles
(pwm\_settings.json)}\label{duty-cycles-pwm_settings.json}}

This file contains the user duty cycle settings for the pwm duty cycle
settings for the fan and the plant light. The keys are fan and light and
the inner keys is duty\_cycle. The user changes the duty cycle
individually for light and fan on the UI advanced settings page. The
duty cycle can be cahanged in units of 10 from 0 to 100 \% duty cycles.

\hypertarget{sensor-log-sensor_log.json}{%
\subsubsection{Sensor Log
(sensor\_log.json)}\label{sensor-log-sensor_log.json}}

This file contains the sensor measurements of the controller taken in
the all cycles including keys of sunlight, soil\_moisture, temperature
and humdiity with the primary key of a datetime string. It is logged
from controller\_main and it cannot be changed or deleted from the UI.
Sunlight units are unscaled from 0 to 120000 lumens / square meter,
temperature is between 0 and 100 degrees F, humidity is between 0 and
100\% RH and soil\_moisture is inverted and then scaled between 0 and
100 in this log. These are essentially exact readings and not converted
to value buckets for the ML!

\hypertarget{temperature-settings-temp_format.json}{%
\subsubsection{Temperature Settings
(temp\_format.json)}\label{temperature-settings-temp_format.json}}

This file contains the temperature scale in \textbf{Fahrenheit or
Celsius}. It could be any value in fahreneheit or celsius. This file can
be changed in the User interface advanced settings tab. Currently, this
file only changes what the temperature in the weather page in the UI is
displayed as.

\hypertarget{utilities-utilities.py}{%
\subsubsection{Utilities (utilities.py)}\label{utilities-utilities.py}}

This contains utilities to translate actions to numbers: e.g.~off is 0,
low is 2 and high is 4. These functions are called in the
controlelr\_main to convert for units for the peripheral changes which
only takes in values from 0 to 4.

\end{document}
